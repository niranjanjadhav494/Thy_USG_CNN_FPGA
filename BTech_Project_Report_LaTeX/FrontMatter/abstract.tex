\thispagestyle{plain}

\chapter*{Abstract}
\addcontentsline{toc}{chapter}{Abstract}

\vspace{2\baselineskip}
\begin{justify}
\noindent
The thyroid gland is a vital endocrine (hormone-producing) gland. It plays a major role in the metabolism, growth, and development of the human body. There are several invasive tests to detect thyroid conditions that involve blood tests or incisions. This is not suitable for all age groups, as it can be more risky for small children and elderly patients. Thyroid nodules present a common medical challenge, requiring accuracy as benign(non-cancerous) or malignant(cancerous) for effective treatment. This project aims to develop a non-invasive method for diagnosing thyroid nodules using Ultrasound Images. The TIRADS(Thyroid Imaging Reporting and Data System) based database has been used to develop a non-invasive technique for thyroid nodule classification. This project aims at developing a non-invasive tool for Thyroid nodule classification with the use of machine learning models. The Convolutional Neural Network (CNN) models have been adopted since these models are proven to be effective in image analysis tasks. While developing the tool for classification, an FPGA-based deployment is considered to accelerate the prediction task of the CNN model. Thus, this task involves the development of the architecture-level design of the CNN model at the hardware level, designing data and control units for the entire set of operations. The approach proposed in this project is scalable and efficient in hardware implementation, making it suitable for deployment in medical diagnosis.
\vspace{1.5\baselineskip}

\noindent
\textbf{Keywords}: CNN, FPGA, Thyroid Nodule, TIRADS
\end{justify}