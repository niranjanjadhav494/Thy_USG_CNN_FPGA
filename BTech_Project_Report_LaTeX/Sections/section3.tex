%\documentclass[../main.tex]{subfiles}

\setlength{\parindent}{2em}
\chapter{Research Gap \& Problem Statement}
\section{Research Gap}
    \noindent Extensive research has been done on FPGA-based hardware accelerators, but some areas are still not fully explored.

    \begin{itemize}
        \item \textbf{HLS Approach}: A lot of work aimed at developing an FPGA-based hardware acceleration scheme, which adopted an HLS(High Level Synthesis) methodology. This is a good approach for faster development of hardware-level design, yet it may suffer from inefficient design if not used properly.

        \item \textbf{Standard Model/Dataset}: It is observed that the models developed on HDL-based methodology also use very commonly used datasets in machine learning, such as the MNIST handwritten digits dataset and CIFAR10, as well as prevalent CNN models such as the LeNet.
    \end{itemize}
\section{Problem Statement}
\noindent
The HDL-based design methodology has been kept at the centre of developing an FPGA hardware accelerator scheme, allowing for better control of the hardware being developed. Thus, this project aims at a systematic approach with the following objectives:
\begin{enumerate}
    \item Design a CNN Model for Thyroid Nodule Classification with \textgreater 90\% Accuracy.
    \item Develop a Hardware Accelerator Architecture for the CNN Model.
    \item Design an Accurate Compute Unit for Hardware-Level Operations.
    \item Design Data and Control Logic for CNN Layers at RTL Level and System Integration.
    \item Develop a GUI for the Hardware Accelerator scheme.
\end{enumerate}
\newpage